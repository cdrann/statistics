\documentclass{article}
\usepackage[utf8]{inputenc}
\usepackage[russian]{babel}
\usepackage{amsmath}
\usepackage{amssymb}
\usepackage{color}
\usepackage{natbib}
\usepackage{graphicx}
\usepackage[dvipsnames]{xcolor}
\definecolor{light-gray}{rgb}{0.8,0.8,0.8}

\title{Расчетное задание по математической статистике}
\author{Редько Анна, \bfвыборка 18: $\alpha = 2, \sigma^2 = 0.7, \varepsilon = 0.18$}

\begin{document}
\maketitle

\colorbox{light-gray}{Задание 1.}
По числовой выборке объема $50$ из нормальной совокупности с параметрами $\alpha$ и $\sigma^2$ (первая выборка) построить доверительные интервалы уровня доверия $1 - \varepsilon$ для параметра:

\underline{\bf $\alpha$, если $\sigma^2$ известно:}

Дана выборка $X = (X_1, \dots, X_n) \sim \Phi_{\alpha, \sigma^2}$. Тогда, из {\bf теоремы о свойствах выборок из нормального распределения}, 
\[\frac{\bar X - \alpha}\sigma \sqrt n \sim \Phi_{0, 1}\]
Из таблиц стандартного нормального распределения находим число $q > 0$ такое, что $\Phi_{0,1}(-q) = \dfrac{\varepsilon}{2}$. Это значит, что 
\[ P\left(-q < \frac{\overline{X} - \alpha}\sigma \sqrt n < q \right) = \Phi_{0,1}(q) - \Phi_{0,1}(-q) = 1 - \varepsilon \]
Это соотношение эквивалентно тому, что 
\[ P\left(\bar X - \dfrac{q\sigma}{\sqrt{n}} < \alpha < \bar X + \dfrac{q\sigma}{\sqrt{n}} \right) = 1 - \varepsilon\]
После подсчетов(см. t1a.R) получили доверительный интервал: 
\colorbox{Lavender}{$(1.78278, 2.10006)$}.

\underline{\bf $\alpha$, если $\sigma^2$ неизвестно:}

По {\bf следствию из теоремы о свойствах выборок из нормального распределения} случайная величина  
\[\frac{\bar X - \alpha}{S} \sqrt {n-1} \sim T_{n-1}\]
распределена по закону Стьюдента с $n-1$ степенью свободы. Подставим $S^2 = \dfrac{n - 1}{n}S_0^2$, тогда 
\[\frac{\bar X - \alpha}{S_0} \sqrt {n} \sim T_{n-1}\]
Из таблиц распределения $T_{n-1}$ находим q такое, что $T_{n-1}(-q) = \dfrac{\varepsilon}{2}.$
Тогда 
\[P\left(-q < \frac{\bar X - \alpha}{S_0} \sqrt {n} < q\right) = T_{n-1}(q) - T_{n-1}(-q) = 1 - \varepsilon, \]
Это соотношение эквивалентно тому, что
\[P \left( \bar X - \dfrac{q S_0}{\sqrt{n}} < \alpha < \bar X + \dfrac{q S_0}{\sqrt{n}}  \right) = 1 - \varepsilon\]
Здесь несмещенную дисперсию вычисляем как 
\[S_0 = \dfrac{1}{\sqrt{n - 1}} \sum_{i=1}^n \left(X_i - \bar X \right).\]
После подсчетов(см. t1b.R) получили доверительный интервал: 
\colorbox{Lavender}{$(1.794088, 2.088752)$}.

\underline{\bf $\sigma^2$, если $\alpha$ известно:}

Случайные величины 
$\frac{X_i - \alpha}\sigma, i = 1, \dots, n$, независимы, и имеют стандартное нормальное распределение, поэтому 
\[\sum_{i=1}^n \left(\dfrac{X_i - \alpha}{\sigma}\right)^2 \sim \chi_n^2.\]
Из таблиц распределения $\chi_n^2$ находим $q_1$ и $q_2$ такие, что $\chi_n^2(q_1) = \dfrac{\varepsilon}{2}$, $\chi_n^2(q_2) = 1 - \dfrac{\varepsilon}{2}$. Тогда 
\[P\left(q_1 < \sum_{i=1}^n \dfrac{(X_i - \alpha)^2}{\sigma^2} < q_2 \right) =  \chi_n^2(q_2) - \chi_n^2(q_1) = 1 - \varepsilon\]
Это соотношение эквивалентно тому, что
\[P\left(  \dfrac{\sum_{i=1}^n(X_i - \alpha)^2}{q_2} < \sigma^2 < \dfrac{\sum_{i=1}^n(X_i - \alpha)^2}{q_1} \right) = 1 - \varepsilon\]
После подсчетов(см. t1c.R), получили, что доверительный интервал: 
\colorbox{Lavender}{$(0.7773202, 0.4528119)$}.

\underline{\bf $\sigma^2$, если $\alpha$ неизвестно:}

Из {\bf теоремы о свойствах выборок из нормального распределения}, 
$\dfrac{n S^2}{\sigma^2} \sim \chi_{n - 1}^2$. 
Из таблиц распределения $\chi_{n - 1}^2$ находим $q_1$ и $q_2$ такие, что $\chi_{n - 1}^2(q_1) = \dfrac{\varepsilon}{2}$ и $\chi_{n - 1}^2(q_2) = 1 - \dfrac{\varepsilon}{2}$.
Тогда 
\[P \left(q_1 < \dfrac{n S^2}{\sigma^2} < q_2 \right) = \chi_{n - 1}^2(q_2) - \chi_{n - 1}^2(q_1) = 1 - \varepsilon.\]
Это соотношение эквивалентно тому, что
\[P \left(\dfrac{n S^2}{q_2} < \sigma^2 < \dfrac{n S^2}{q_1}\right) = 1 - \varepsilon.\]
После подсчетов(см. t1d.R), получили, что доверительный интервал: 
\colorbox{Lavender}{$(0.458274, 0.7910913)$}.

\colorbox{light-gray}{Задание 2.}
По данным числовым наблюдениям (вторая выборка объема $30$) проверить основную гипотезу о равномерности распределения с помощью а) критерия Колмогорова, б) критерия $\chi^2$ (асимптотического размера $\varepsilon$). Построить график эмпирической функции распределения. Найти реально достигнутый уровень значимости.

\underline{\bf используя критерий Колмогорова:}

Функция распределения K(y) называется функцией Колмогорова, она абсолютно непрерывна; 
для нахождения ее значений имеются таблицы.

Перейдем к построению критерия.

Пусть $X \sim F$ и проверяются гипотезы $H_1$ : $F = U_{0, 1}$ против $H_2$ : $F \neq U_{0, 1}$, где $U_{0, 1}$ непрерывна. 

Наша задача: построить асимптотический критерий уровня $1 - \varepsilon$.

Для начала вычислим величину $D_n$ в предположении, что верна гипотеза $H_1$, т. е.
$F = U_{0, 1}$:
\[D_n = \sup_{t} |F_n^*(t)-U_{0,1}(t)|\]
\[D_n = 0.887\]
\[\sqrt(n) D_n = 4.858299\]
Тут $F_n^*$ - эмперическая функция распределения, $U_{0,1}$ - теоретическая функция распределения. В силу теоремы Колмогорова, при больших $n$ функция распределения случайной
величины $\sqrt{n}D_n$ мало отличается от $K(y)$, поэтому заранее по таблицам функции
Колмогорова мы можем найти такое число q > 0, что $K(q) = 1 - \varepsilon$. Нашли, что $q = 1.1$.
Следовательно, если верна $H_1$, то $P_1 \left(\sqrt{n} D_n < q \right) \simeq K(q) = 1 - \varepsilon$. 
Поэтому мы будем отвергать гипотезу $H_1$, если окажется, что $\sqrt{n} D_n \geqslant q$ ,
т. е. если расхождение между эмпирической и гипотетической функциями распределения достаточно велико. В нашем случае это именно так, ведь $4.858299 > 1.1$, \colorbox{Lavender}{гипотезу $H_1$ мы отвергаем}. 
Ясно, что при этом
\[\beta_1 = P_1 \left( \sqrt{n} D_n \geqslant q)\right) = 1 - P_1 \left(\sqrt{n} D_n < q\right) \simeq 1 - K(q) = \varepsilon = 0.18\] 
Критическое множество для построенного критерия: 
\[K = {(X_1, \dots, X_n) \in \mathbb{R}^n : \sqrt{n}D_n \geqslant q}. \]
Достигаемый уровень значимости критерия Колмогорова равен:
\[\alpha^* = 2 \sum_{k = 1}^\infty (-1)^{k+1} e^{-2k^2 n D_n^2} < 0.1\].

\underline{\bf используя критерий $\chi^2$:}  

Пусть $X \sim F$ и проверяются гипотезы $H_1$ : $F = U_{0, 1}$ против $H_2$ : $F \neq U_{0, 1}$.
По-прежнему наша задача состоит в построении асимптотического критерия уровня
$1 - \varepsilon$. В предположении, что $X \sim U_{0, 1}$, разобьем область возможных значений $X_1 = [0, 1)$ на k непересекающихся промежутков(здесь k ищем по формуле Стеджеса $k = \lfloor log_2 30 \rfloor + 1 = 4 + 1 = 5$):
\[ P_1(X_1 \in \Delta_1 \cup \dots \cup  \Delta_k) = 1,\]
где $\Delta_i$ имеет вид $\Delta_i = \left[a_i; b_i\right), i = 1, \dots, k.$

Пусть $\nu_i$ - число наблюдений, попавших в $\Delta_i, i = 1, \dots, k, \nu_1 + \dots + \nu_k = n.$

Обозначим также 
\[p_i = P_1\left(X_1 \in \Delta_i\right) = U_{0, 1}(b_i) - U_{0, 1}(a_i), i = 1, \dots, k.\]

Из закона больших чисел следует, что 
\[ \frac{\nu_i}{n} \xrightarrow{\mathbb{P}} p_i, n \rightarrow \infty,\]
при каждом $i$, если верна $H_1$. 
В качестве меры близости совокупностей 
${\nu_1 / n, \dots, \nu_k / n}$ и ${p_1, \dots, p_k}$ предлагается использовать величину
\[\Psi_n = n \sum_{i=1}^k \dfrac{1}{p_i} \left(\dfrac{\nu_i}{n} - p_i\right)^2 = \sum_{i=1}^k \dfrac{\left(\nu_i - n p_i \right)^2}{n p_i}.\]
Т.к. $p_i = 0.2$ $\forall i$, можем представить $\Psi_n$ как
\[\Psi_n = \dfrac{1}{n p_i}\sum_{i=1}^k \left(\nu_i - n p_i \right)^2.\]
\[\Psi_n = 4.666667\] 

{\bf Теорема Пирсона.} Если $0 < p_i < 1$ при всех $i = 1, \dots, k$, то для любого $y > 0$
\[P_1(\Psi_i < y) \rightarrow \chi_{k - 1}^2(y), n \rightarrow \infty .\]

Займемся построением критерия. Найдем число $q$ такое, что $\chi_{k-1}^2(q) = 1 - \varepsilon.$ Получим $q = 6.268116.$ 

Если верна гипотеза $H_1$, то с вероятностью, близкой к $1 - \varepsilon$, значение случайной величины $\Psi_n$ должно быть меньше $q$. Поэтому мы отвергаем гипотезу, если $\Psi_n \geqslant q$, и принимаем ее в противном случае. В нашем случае $\Psi_n < q$, а значит \colorbox{Lavender}{гипотезу $H_1$ мы принимаем}. 

Это значит, что мы принимаем $H_1$, если нет явного противоречия этой гипотезы с наблюденными значениями. Критическое множество в данном случае: 
\[K = {(X_1, \dots, X_n) : \Psi_n \geqslant q}.\]

Для вероятности ошибки первого рода имеем 
\[\beta_1 = P_1 \left( \Psi_n \geqslant q \right) = 1 - P_1 \left(\Psi_n < q\right) \simeq 1 - \chi_{k - 1}^2(q) = \varepsilon = 0.18\] 

Достигаемый уровень значимости критерия $\chi^2$ равен:
\[\varepsilon^* =  0.323239.\]


\colorbox{light-gray}{Задание 3.}
По данным двум выборкам из нормальных совокупностей (первые $20$ и следующие $30$ элементов первой выборки) проверить, с помощью критериев размера $\varepsilon$, гипотезу:

\underline{\bf о совпадении дисперсий при неизвестных средних:}

Имеем две независимые выборки
\[X = \left(X_1,\dots, X_n \right) \sim \Phi_{\alpha_1, \sigma_1^2}\]
\[Y = \left(Y_1,\dots, Y_m \right) \sim \Phi_{\alpha_2, \sigma_2^2}\]

В этом пункте проверяем гипотезу $H_1: \sigma_1^2 = \sigma_2^2$ против $H_2: \sigma_1^2 \neq \sigma_2^2$. По условию $\varepsilon = 0.18$ и пусть 
\[ \bar X  = \dfrac{1}{n} \sum_{i=1}^n X_i, \] 
\[ S_{X}^2 = \dfrac{1}{n} \sum_{i=1}^n \left(X_i - \bar X\right)^2, \]
\[ \bar Y  = \dfrac{1}{m} \sum_{i=1}^m Y_i, \] 
\[ S_{Y}^2 = \dfrac{1}{m} \sum_{i=1}^m \left(Y_i - \bar Y\right)^2, \]

По теореме о свойствах выборок из нормального распределения 
\[\dfrac{n S_{X}^2}{\sigma_1^2} \sim \chi_{n - 1}^2,\]
\[\dfrac{m S_{Y}^2}{\sigma_2^2} \sim \chi_{m - 1}^2,\]

причем эти случайные величины независимы, поскольку построены по независимым
выборкам. Из них можно построить случайную величину, имеющую распределение
Фишера:
\[\dfrac{1}{n - 1} \dfrac{n S_{X}^2}{\sigma_1^2} : \dfrac{1}{m - 1} \dfrac{m S_{Y^2}}{\sigma_2^2} = \dfrac{n (m - 1) \sigma_2^2 S_x^2}{m (n - 1) \sigma_1^2 S_y^2} \sim F_{n - 1, m - 1}. \]

Если верна гипотеза $H_1$, т.е. $\sigma_1^2 = \sigma_2^2$, то 
\[ \eta = \dfrac{n (m - 1) S_{X}^2}{m (n - 1) S_{Y}^2} \sim F_{n - 1, m - 1}\]
\[ \eta = 1 \]

С помощью таблиц распределения $F_{n - 1, m - 1}$ можно найти числа $q_1$ и $q_2$ такие, что $F_{n - 1, m - 1}(q_1) = \varepsilon / 2, F_{n - 1, m - 1}(q_2) = 1 - \varepsilon / 2.$ Получили, что $q_1 = 0.5523387$, $q_2 = 1.726676$. Тогда 
\[P_1\left(q_1 < \eta < q_2\right) = F_{n - 1, m - 1}(q_2) - F_{n - 1, m - 1}(q_1) = 1 - \varepsilon = 0.82.\]
Поэтому логично отвергать $H_1$, если $\eta \not\in (q_1, q_2)$; вероятность такого события равна в точности $\varepsilon$, если верна $H_1$.

В нашем случае $1 \in (0.5523387, 1.726676)$, так что \colorbox{Lavender}{гипотезу $H_1$ мы принимаем}.

Здесь 
\[K =  {(X_1, \dots, X_n, Y_1, \dots, Y_m) : \eta \notin (q_1, q_2)}.\]

\underline{\bf о совпадении средних, если известно, что неизвестные дисперсии совпадают:}

Имеем две независимые выборки
\[X = \left(X_1,\dots, X_n \right)\]
\[Y = \left(Y_1,\dots, Y_m \right)\]

По условию, дисперсии совпадают: $\sigma_1^2=\sigma_2^2=\sigma^2$, при этом $\sigma^2$ неизвестна.

Необходимо проверить гипотезу $H_1: \alpha_1 = \alpha_2$ против $H_2: \alpha_1 \neq \alpha_2$.

Воспользуемся распределением Стьюдента. В силу того, что $\bar X$ и $\bar Y$ независимы, и
\[\bar X \sim \Phi_{\alpha_1, \sigma^2/n}\]
\[\bar Y \sim \Phi_{\alpha_2, \sigma^2/m},\]

имеем 
\[\bar X - \bar Y \sim \Phi_{\alpha_1 - \alpha_2, \sigma^2(1/n+1/m)}\]

После стандартизации:
\[\dfrac{\bar X - \bar Y - (\alpha_1 - \alpha_2)}{\sqrt{\sigma^2\left(\dfrac{1}{n} + \dfrac{1}{m}\right)}} \sim \Phi_{0, 1}\]

Далее, по свойству распределения хи-квадрат, 
\[\dfrac{n S_{X}^2}{\sigma^2} + \dfrac{m S_{Y}^2}{\sigma^2} \sim \chi_{n + m - 2}^2,\]

эта случайная величина не зависит от $\bar X - \bar Y$. Таким образом:
\[\dfrac{\bar X - \bar Y - (\alpha_1 - \alpha_2)}{\sigma \sqrt{\dfrac{1}{n}+ \dfrac{1}{m}}} : \sqrt{\dfrac{1}{n + m - 2} \dfrac{n S_{X}^2 + m S_{Y}^2}{\sigma^2}} \sim T_{n+m-2}.\]

Если верна гипотеза $H_1$, то $\alpha_1 - \alpha_2 = 0$
\[\psi = \dfrac{\bar X - \bar Y}{\sqrt{\dfrac{1}{n} + \dfrac{1}{m}} \sqrt{\dfrac{n S_{X}^2 + m S_{Y}^2}{n + m - 2}}} \sim T_{n + m - 2}.\]
\[\psi = 1.754003\]
Из таблиц распределения $T_{n + m - 2}$ находим $q$ такое, что $T_{n + m - 2}(-q) = \dfrac{\varepsilon}{2}.$ Нашли, что $-q = -1.360585.$

Тогда 
\[P_1(-q < \psi < q) = T_{n + m - 2}(q) - T_{n + m - 2}(-q) = 1 - \varepsilon.\]

В нашем случае, получаем, что $1.754003 \not\in (-1.360585, 1.360585)$. \colorbox{Lavender}{Гипотеза $H_1$ не принимается}. 

Следовательно, выбрав 
\[K = {(X_1, \dots, X_n, Y_1, \dots, Y_m) : |\psi| \geqslant q},\]

мы будем иметь 
\[ \beta_1 = P_1((X_1, \dots, X_n, Y_1, \dots, Y_m) \in K) = \varepsilon = 0.18.\]

\end{document}